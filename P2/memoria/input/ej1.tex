\section{Comunicación entre procesos MPI.}
En este ejercicio se nos pide hacer un programa con distintas funcionalidades,
de forma que interaccione con el usuario para preguntarle qué funcionalidad quiere
ejecutar. Decidimos resolver con una estructura similar a la de llamada a procedimiento
remoto.

Vamos identificando sub-problemas y resolvemos secuencialmente:
\begin{enumerate}
    \item Decide la estructura de la solución.
    \item No permite ejecutar si no se ha invocado el programa con cuatro procesos.
    \item Haz que cada proceso ejecute su rutina correspondiente.
    \item Permite que el usuario pueda elegir una opción de entre las posibles que se nos solicitan (entero en el rango [0,4]).
    \item Gestiona los posibles errores correspondientes a la entrada.
    \item Implementa la función cero (finalizar) como un \texttt{MPI\_Abort}.
    \item Para la función uno, pasar a mayúsculas una cadena de texto:
    \begin{enumerate}
        \item Permite al usuario introducir una cadena de texto.
        \item Gestiona errores para la entrada de texto.
        \item Crea un stub para capitalizar una cadena de texto el cual invoque el manejador de la función uno.
        \item Implementa el trabajador toupper.
        \item Comunica la rutina del stub con el trabajador toupper.
    \end{enumerate}
    \item Para la función dos, calcular la suma y su cuadrado dada una lista de números reales:
    \begin{enumerate}
        \item Define una estructura que almacena el resultado de la operación.
        \item Implementa el trabajador realsumroot, el stub y el handler.
        \item Imprime el resultado con formato.
    \end{enumerate}
    \item Para la función tres, calcular la suma de los caracteres de una cadena de texto:
    \begin{enumerate}
        \item Implementa el trabajador intsum, el stub y el handler.
    \end{enumerate}
    \item Permite al trabajador toupper recibir cadenas de texto de longitud variable.
    \item Crea una interfaz de texto para que el usuario tenga información de lo que se está ejecutando y del estado en que se cuentra.
    \item Implementa la función cero como un mensaje que se envía a todos los procesos para apagarlos.
    \item Implementa la función cuatro, realizar consecutivamente las funciones 1-3.
\end{enumerate}

Adecuamos la paralelización de los procesos a la naturaleza de la entrada del programa, tamaño de los datos, número de procesos preestablecido, etc. Si las necesidades fuesen otras, adaptaríamos el código
a estas. Mientras que se pueda, primamos la simplicidad.

Podemos ejecutar este ejercicio navegando hasta el directorio asociado y ejecutando \texttt{make run}.
