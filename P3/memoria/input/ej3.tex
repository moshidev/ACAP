\section{Cálculo del Índice de Jaccard.}

Conociendo la intersección de los dos conjuntos podemos conocer este índice.
La mejor forma que encontramos para calcular la intersección de dos conjuntos
es la de utilizar un \textit{hashset}. Escogemos la implementación \textit{khash}.

Paralelizamos la carga de los datos en dos hebras, una por conjunto. De no ser así
ocurrirían condiciones de carrera al no ser la implementación escogida \textit{thread-safe}.

Una vez tenemos los dos conjuntos en memoria, paralelizamos el tamaño de la intersección
entre tantas hebras como queramos. Podemos hacer esto porque la lectura no modifica la
estructura de datos.

Ejecutamos con \texttt{make run}.

\subsection{Trabajo futuro.}

Podríamos paralelizar la carga de datos si permitiésemos que la estructura de datos fuese
reentrante. Para ello tal vez podríamos encontrar una solución que usase un cerrojo de espera
ocupada por cada \textit{bucket} junto a un \textit{mutex} que sincronizase el redimensionado
de la estructura de datos cuando fuese necesario (similar al problema de los lectores-escritores).