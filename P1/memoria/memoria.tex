% Adapted from Atanasio Rubio Gil's https://gitlab.com/Groctel/aqademia/-/blob/main/demo/demo_aqademia.tex

\documentclass[10pt, a4paper]{aqademic}

% Language and input encoding

\usepackage[spanish]{babel}

% Document settings

\usepackage[type=CC, modifier=by-nc-sa, version=4.0]{doclicense}
\usepackage{graphicx}
	\graphicspath{{img/}}
\usepackage{multirow}
\usepackage{adjustbox}
\usepackage{float}

\author{Daniel Pedrosa Montes}
\title{Arquitectura y Computación de Altas Prestaciones}

% Document composition

\begin{document}

\AqMaketitle[%
	cover    = identidad_ugr,
    subtitle = Manejo de MPI,
    url      = https://github.com/moshidev,
    date     = marzo del 2023 
]

\tableofcontents

\chapter{Resolución de los ejercicios propuestos.}
    Para las ejecuciones en nuestro ordenador local utilizamos \texttt{clang 14.0} sin flags
    adicionales de compilación además de los necesarios y para las ejecuciones en \texttt{atcgrid.ugr.es}
    utilizamos \texttt{gcc 9.2.0}.

    \section{Búsqueda del máximo de un vector sin uso de mecanismos de exclusión mutua explícita.}

Resolvemos el problema dividiendo los trozos del vector entre las distintas hebras. Cada
hebra almacena el máximo encontrado en su trozo asignado. Después de esperar a que todas
las hebras terminen buscamos el máximo de entre estos valores devueltos.

Ejecutamos con \texttt{make run}.
    \pagebreak
    \section{Paralelización de un algoritmo de procesamiento digital de imágenes.}

    \pagebreak
    \section{Cálculo del Índice de Jaccard.}

Conociendo la intersección de los dos conjuntos podemos conocer este índice.
La mejor forma que encontramos para calcular la intersección de dos conjuntos
es la de utilizar un \textit{hashset}. Escogemos la implementación \textit{khash}.

Paralelizamos la carga de los datos en dos hebras, una por conjunto. De no ser así
ocurrirían condiciones de carrera al no ser la implementación escogida \textit{thread-safe}.

Una vez tenemos los dos conjuntos en memoria, paralelizamos el tamaño de la intersección
entre tantas hebras como queramos. Podemos hacer esto porque la lectura no modifica la
estructura de datos.

Ejecutamos con \texttt{make run}.

\subsection{Trabajo futuro.}

Podríamos paralelizar la carga de datos si permitiésemos que la estructura de datos fuese
reentrante. Para ello tal vez podríamos encontrar una solución que usase un cerrojo de espera
ocupada por cada \textit{bucket} junto a un \textit{mutex} que sincronizase el redimensionado
de la estructura de datos cuando fuese necesario (similar al problema de los lectores-escritores).
    
    \appendix
    \chapter{Fusión de bucles en el cálculo de Pi mediante distintos métodos.}
        Podemos identificar que ambos algoritmos utilizan un mismo número de pasos.
Como la localidad espacial no entra en juego para ninguno de estos podemos
acelerar la ejecución fusionando ambos bucles en uno. Modificamos el código,
ejecutamos las pruebas sobre \texttt{atcgrid} y obtenemos los resultados que
mostramos en la figura A.1.

\begin{figure}[H]
    \centering
    \includegraphics[width=16cm]{ap.png}
    \caption{Relación segundos / número de procesos. Observamos cómo los tiempos de ejecución para
    el algoritmo en que fusionamos ambos algoritmos son menores que la suma de los dos algoritmos ejecutándose
    por separado. Esto se debe a que se divide entre dos el tiempo que el procesador gasta controlando el bucle
    \texttt{for}.}
\end{figure}
    
\bibliographystyle{plain}
\bibliography{references}

\end{document}
