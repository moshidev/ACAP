\section{Cálculo del producto escalar.}

Partimos del ejemplo 4 del primer seminario para implementar un programa
que nos permita realizar el cálculo del producto escalar. Para ello seguimos
los siguientes pasos, comprobando en cada uno de ellos que todo funciona según esperamos,
para diferente número de tareas y para diferente tamaño.

\begin{enumerate}
    \item Aisla el código cuya función es realizar la sumatoria de un vector en una función llamada `job`.
    \item Haz que el proceso MASTER participe en el cálculo, dividiendo la carga de trabajo entre todos los procesos.
    \item Haz que la recepción de los resultados de los workers se reciba de forma asíncrona, de forma que el MASTER pueda ir trabajando a la vez sin que esto suponga que los workers tengan que esperar a que termine este.
    \item Aisla el código del MASTER que se comunica con los workers en una función `scalarProduct`.
    \item Modifica la función job, worker y la que acabamos de crear de forma que haga el producto escalar.
\end{enumerate}

El ejecutable resultante tiene la misma interfaz que la del ejemplo 4.
Si, por ejemplo, decidimos un tamaño de 5 realizará la multiplicación $(0,1,2,3,4) \cdot (0,1,2,3,4)$.
Entendemos que la entrada de datos personalizados se escapa de los objetivos de la entrega.

Para compilar y ejecutar el ejercicio utilizamos \texttt{make run}.
